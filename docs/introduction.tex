\chapter{Introduction}
Perhaps, the sinking of the RMS Titanic is the most infamous shipwreck 
in history. According to the 
\href{https://en.wikipedia.org/wiki/Titanic}{Wikipedia}, 
the RMS Titanic was the largest ocean liner in service at the time. It
had advanced safety features, such as watertight compartments and 
remotely activated watertight doors. The ship was widely considered 
"unsinkable". However, the Titanic sank in the early morning of 15 April
1912 in the North Atalntic Ocean during her maiden voyage from 
Southampton to New York City. There were an estimated 2224 people on
board when the ship collided with an iceberg
\cite{titanic-wikipedia},\cite{sinking-of-the-titanic-wikipedia}.

In accordance with existing practice, Titanic's lifeboat system was 
designed to ferry passengers to nearby rescue vessels, not to hold 
everyone on board simultaneously; therefore, with the ship sinking 
rapidly (the ship had sank in 2 hours and 40 minutes) and help still 
hours away, there was no safe refuge for many of the passengers and 
crew with only 20 lifeboats. Poor management of the evacuation meant 
many boats were launched before they were completely full 
\cite{sinking-of-the-titanic-wikipedia}.

The shipwreck resulted in the deaths of more than 1500 people, makng it 
one of the deadliest in history \cite{sinking-of-the-titanic-wikipedia}.

Without a doubt, there was an element of luck involved in surviving, but,
possibly, some groups of people were more likely to survive than others.
The \href{https://www.kaggle.com/c/titanic}{Titanic ML competition on 
Kaggle} offers participants to predict which of the passengers survived 
the shipwreck using passenger data\cite{titanic-ml-competition}.

In this report I'm going to describe my solution of the
\href{https://www.kaggle.com/c/titanic}{Titanic ML competition's} task.
My workflow will be based mostly on the
\href{https://github.com/ageron/handson-ml/blob/master/ml-project-checklist.md}
{"Machine Learning project checklist"} from the book \cite{hands_on_ml}.
I really appreciate this book and highly recommend reading it to anyone 
starting to learn about machine learning. 

Dozens of articles dedicated to this competition and hundreds of solutions 
of this task are available in the Internet. Therefore, I won't cite to 
all materials seen, but I'll try to give several useful refrences.
In exploratory analysis, I relied a lot on the tutorial \cite{habr_titanic} 
and borrowed several ideas from it.