\chapter{Preliminary Analysis}

\section{Shape of the dataset}
The dataset contains:
\begin{itemize}
	\item 891 rows, 
	\item 12 columns.
\end{itemize}


\section{First rows of the dataset}
An excerpt from the dataset is presented in the table
\ref{table:excerpt_from_dataset}.

\begin{table}[!ht]
	\centering
	\caption{Excerpt from the dataset}
	\resizebox{\textwidth}{!}{
	\begin{tabular}{|r|r|r|r|r|r|r|r|r|r|r|r|r|}
		\hline
		           & \textbf{PassengerId} & \textbf{Survived} & \textbf{Pclass} & \textbf{Name}                                     & \textbf{Sex} & \textbf{Age} & \textbf{SibSp} & \textbf{Parch} & \textbf{Ticket}  & \textbf{Fare} & \textbf{Cabin} & \textbf{Embarked} \\ \hline
		\textbf{0} & 1                    & 0                 & 3               & Braund, Mr. Owen Harris                           & male         & 22.0         & 1              & 0              & A/5 21171        & 7.2500        & NaN            & S                 \\ \hline
		\textbf{1} & 2                    & 1                 & 1               & Cumings, Mrs. John Bradley (Florence Briggs Th... & female       & 38.0         & 1              & 0              & PC 17599         & 71.2833       & C85            & C                 \\ \hline
		\textbf{2} & 3                    & 1                 & 3               & Heikkinen, Miss. Laina                            & female       & 26.0         & 0              & 0              & STON/O2. 3101282 & 7.9250        & NaN            & S                 \\ \hline
		\textbf{3} & 4                    & 1                 & 1               & Futrelle, Mrs. Jacques Heath (Lily May Peel)      & female       & 35.0         & 1              & 0              & 113803           & 53.1000       & C123           & S                 \\ \hline
		\textbf{4} & 5                    & 0                 & 3               & Allen, Mr. William Henry                          & male         & 35.0         & 0              & 0              & 373450           & 8.0500        & NaN            & S                 \\ \hline
	\end{tabular}}
	\label{table:excerpt_from_dataset}
\end{table}

The \textbf{"PassengerId"} feature is the ID of the passanger. It won't 
help in analysis and will be dropped. Also, there are several missing 
values, and some values are categorical, for example, \textbf{"Pclass"}
and \textbf{"Sex"}.


\section{Data types and missing values}
Table \ref{table:dtypes} contains types of the data in each column and
numbers of non-null values. Table \ref{table:missing_values} contains 
numbers of missing values in each column.

\begin{table}[!ht]
	\centering
	\caption{Data types and non-null counts}
	\begin{tabular}{|l|l|l|l|}
		\hline
		\textbf{\#} & \textbf{Column} & \textbf{Non-Null Count} & \textbf{Dtype} \\ \hline
		\textbf{0}  & PassengerId     & 891 non-null            & int64          \\ \hline
		\textbf{1}  & Survived        & 891 non-null            & int64          \\ \hline
		\textbf{2}  & Pclass          & 891 non-null            & int64          \\ \hline
		\textbf{3}  & Name            & 891 non-null            & object         \\ \hline
		\textbf{4}  & Sex             & 891 non-null            & object         \\ \hline
		\textbf{5}  & Age             & 714 non-null            & float64        \\ \hline
		\textbf{6}  & SibSp           & 891 non-null            & int64          \\ \hline
		\textbf{7}  & Parch           & 891 non-null            & int64          \\ \hline
		\textbf{8}  & Ticket          & 891 non-null            & object         \\ \hline
		\textbf{9}  & Fare            & 891 non-null            & float64        \\ \hline
		\textbf{10} & Cabin           & 204 non-null            & object         \\ \hline
		\textbf{11} & Embarked        & 889 non-null            & object         \\ \hline
	\end{tabular}
	\label{table:dtypes}
\end{table}

\begin{table}[!ht]
	\centering
	\caption{Number of missing values in each column}
	\begin{tabular}{|l|l|l|}
		\hline
		\textbf{\#} & \textbf{Column}   & \textbf{Number of missing values} \\ \hline
		\textbf{0}  & PassengerId       & 0                                 \\ \hline
		\textbf{1}  & Survived          & 0                                 \\ \hline
		\textbf{2}  & Pclass            & 0                                 \\ \hline
		\textbf{3}  & Name              & 0                                 \\ \hline
		\textbf{4}  & Sex               & 0                                 \\ \hline
		\textbf{5}  & \textbf{Age}      & \textbf{177}                      \\ \hline
		\textbf{6}  & SibSp             & 0                                 \\ \hline
		\textbf{7}  & Parch             & 0                                 \\ \hline
		\textbf{8}  & Ticket            & 0                                 \\ \hline
		\textbf{9}  & Fare              & 0                                 \\ \hline
		\textbf{10} & \textbf{Cabin}    & \textbf{687}                      \\ \hline
		\textbf{11} & \textbf{Embarked} & \textbf{2}                        \\ \hline
	\end{tabular}
	\label{table:missing_values}
\end{table}


\pagebreak
\section{Number of unique values}
Table \ref{table:unique_values} contains numbers of unique values in 
each column.

\begin{table}[!ht]
	\centering
	\caption{Number of unique values in each column}
	\resizebox{\textwidth}{!}{
	\begin{tabular}{|
	>{\columncolor[HTML]{EFEFEF}}l |l|
	>{\columncolor[HTML]{EFEFEF}}l |l|}
		\hline
		\textbf{Column} & \textbf{Number of unique values} & \textbf{Column} & \textbf{Number of unique values} \\ \hline
		Name            & 891                              & Survived        & 2                                \\ \hline
		Sex             & 2                                & Pclass          & 3                                \\ \hline
		Ticket          & 681                              & Age             & 88                               \\ \hline
		Cabin           & 147                              & SibSp           & 7                                \\ \hline
		Embarked        & 3                                & Parch           & 7                                \\ \hline
		PassengerId     & 891                              & Fare            & 248                              \\ \hline
	\end{tabular}}
	\label{table:unique_values}
\end{table}

There are high-cardinality features with object dtype:
\begin{itemize}
	\item Name
	\item Ticket
	\item Cabin
	\item PassengerId
\end{itemize}
This features, possibly, will need special preprocessing. Earlier, I 
noticed that the \textbf{"PassengerId"} feature is the ID of the 
passanger. It won't help in analysis and will be dropped. Features 
\textbf{"Age"} and \textbf{"Fare"} are continuous.


\section{Summary statistics}

\begin{table}[!ht]
	\centering
	\caption{Summary statistics for numerical variable}
	\resizebox{\textwidth}{!}{
	\begin{tabular}{|r|r|r|r|r|r|r|r|}
		\hline
		               & \textbf{PassengerId} & \textbf{Survived} & \textbf{Pclass} & \textbf{Age} & \textbf{SibSp} & \textbf{Parch} & \textbf{Fare} \\ \hline
		\textbf{count} & 891.000000           & 891.000000        & 891.000000      & 714.000000   & 891.000000     & 891.000000     & 891.000000    \\ \hline
		\textbf{mean}  & 446.000000           & 0.383838          & 2.308642        & 29.699118    & 0.523008       & 0.381594       & 32.204208     \\ \hline
		\textbf{std}   & 257.353842           & 0.486592          & 0.836071        & 14.526497    & 1.102743       & 0.806057       & 49.693429     \\ \hline
		\textbf{min}   & 1.000000             & 0.000000          & 1.000000        & 0.420000     & 0.000000       & 0.000000       & 0.000000      \\ \hline
		\textbf{25\%}  & 223.500000           & 0.000000          & 2.000000        & 20.125000    & 0.000000       & 0.000000       & 7.910400      \\ \hline
		\textbf{50\%}  & 446.000000           & 0.000000          & 3.000000        & 28.000000    & 0.000000       & 0.000000       & 14.454200     \\ \hline
		\textbf{75\%}  & 668.500000           & 1.000000          & 3.000000        & 38.000000    & 1.000000       & 0.000000       & 31.000000     \\ \hline
		\textbf{max}   & 891.000000           & 1.000000          & 3.000000        & 80.000000    & 8.000000       & 6.000000       & 512.329200    \\ \hline
	\end{tabular}}
	\label{table:numerical_variable_description}
\end{table}

\begin{table}[!ht]
	\centering
	\caption{Summary statistics for categorical variable}
	\begin{tabular}{|r|r|r|r|r|r|}
		\hline
		\textbf{}       & \textbf{Name}           & \textbf{Sex} & \textbf{Ticket} & \textbf{Cabin} & \textbf{Embarked} \\ \hline
		\textbf{count}  & 891                     & 891          & 891             & 204            & 889               \\ \hline
		\textbf{unique} & 891                     & 2            & 681             & 147            & 3                 \\ \hline
		\textbf{top}    & Braund, Mr. Owen Harris & male         & 347082          & B96 B98        & S                 \\ \hline
		\textbf{freq}   & 1                       & 577          & 7               & 4              & 644               \\ \hline
	\end{tabular}
	\label{table:categorical_variable_description}
\end{table}